%
%  thesis.tex  2014-08-08  Mark Senn
%
%  This is the root file for a simple example thesis.
%  This example can also be used to prepare a dissertation.
%
%  To make a final copy of your thesis put a '%'
%  in front of the \includeonly command and run:
%    latex thesis
%    latex thesis
%    latex thesis
%    bibtex thesis
%    latex thesis
%    latex thesis
%
%  References cited below:
%
%    TM1996 is short for Thesis Manual 1996.
%    ``A Manual for the Preparation of Graduate Theses'',
%    The Graduate School, Purdue University, 1996.
%
%    TM2006 is short for Thesis Manual 2006.
%    ``A Manual for the Preparation of Graduate Theses'',
%    seventh revised edition, The Graduate School, Purdue University, 2006.
%    http://www.purdue.edu/GradSchool/documents/thesis/graduate-thesis-manual.pdf
%
%  Search for "CHANGE" below and change things as necessary.
%  I recommend putting "%%" before any existing lines that
%  need to be changed and adding your new line(s) immediately
%  below the existing lines.
%

% See
%     http://www.ecn.purdue.edu/~mark/puthesis/#Options
% for documentclass options.
% CHANGE NEXT LINE?
\documentclass[cs,thesis,uglyheadings]{puthesis}

% Define "align" environment used in demo-mathematics.tex.
% CHANGE NEXT LINE?
\usepackage{amsmath}

% Define "multicols" environment environment used in demo-multicols.tex.
% CHANGE NEXT LINE?
\usepackage{multicol}

% Define "subfigure" environment used in "demo-figure.tex".
% CHANGE NEXT LINE?
\usepackage{subfigure}

% Title of thesis (used on cover and in abstract).
% The title shown must be the full, official title of the thesis.
% Superscripts and subscripts are not permitted in the title.
% Reference: TM2006, page 26.
% Use \title{Put Title Here} for a one-line title.
% Use \\ to separate lines in multi-line titles.
% Put % at the end of the last line of a title
% to avoid getting an extra space in the abstract.
\title{%
  Using Time Series Models for Defect Prediction\\
  in Software Release Planning%
}

% First author name with first name first is used for cover.
% Second author name with last name first is used for abstract.
% Your full name as it appears in the University records appears
% on the cover.
% Reference: TM2006 pages 26, 29.
\author{James W. Tunnell}{Tunnell, James W.}

% First is long title of degree (used on cover).
% Second is abbreviation for degree (used in abstract).
% Third is the month the degree was (will be) awarded (used on cover
% and in abstract).
% Last is the year the degree was (wlll be) awarded (used on cover
% and in abstract).
% The degree title for all doctoral candidates is ``Doctor of Philosophy''.
% The precise degree names for master's candidates appear in the list of
% ``Degrees Offered'' in the Graduate School bulletin.
% The date is the month and year that the degree is actually awarded.
% (If you have registered for ``degree only'', revise the thesis title
% page to reflect the new date on which the degree is to be awarded.)
% Reference: TM2006 pages 26--27, 30.
\pudegree{Master of Science}{MS}{June}{2015}

% Major professor (used in abstract).
\majorprof{John D. Anvik}

% Campus (used only on cover)
% Reference: TM2006 page 27.
\campus{Ellensburg}


%
% My command definitions not specific to my thesis.
%

% CHANGE NEXT LINE?
\input{mydefs}


%
% My command definitions specific to my thesis.
%

% CHANGE NEXT TWO LINES?
% Let typing "\en" be exactly the same as typing "\ensuremath". 
\let\en=\ensuremath

% CHANGE NEXT TWO LINES?
% Set things up so \margins will show where the margins on the page are.
\newcommand{\margins}{\Repeat{Show where the margins for the page are.}{4}}

% CHANGE NEXT FIVE LINES?
% Define a \ve command with two arguments, so if it called with
%     \ve an
% it will expand to
%     {\en{a_1},~\en{a_2},\ \ldots,~\en{a_{n}}}
\newcommand{\ve}[2]{\en{#1_1},~\en{#1_2},\ \ldots,~\en{#1_{#2}}}


% To LaTeX only some parts of your thesis put the
% names of the parts to include here.  For example,
% \includeonly{front} would only process front.tex.
% \includeonly{front,introduction} would only process
% front.tex and introduction.tex.
% To print the final copy of your thesis put a '%'
% in front of the \includeonly command and run LaTeX
% three times to make sure that all cross-references
% are correct.  Then run BibTeX once and LaTeX twice
% more.
% CHANGE NEXT LINE?
%\includeonly{front,introduction}

\begin{document}

% Start a new volume for your thesis.
% All theses must have at least one volume.
% If your thesis has multiple volumes put another "\volume"
% command between chapters below.
\volume

% Front matter:
%     dedication
%     acknowledgments
%     preface
%     table of contents
%     list of tables
%     list of figures
%     list of symbols
%     list of abbreviations
%     nomenclature
%     glossary
%     abstract
%     publication
%
%  revised  front.tex  2011-09-02  Mark Senn  http://engineering.purdue.edu/~mark
%  created  front.tex  2003-06-02  Mark Senn  http://engineering.purdue.edu/~mark
%
%  This is ``front matter'' for the thesis.
%
%  Regarding ``References'' below:
%      KEY    MEANING
%      PU     ``A Manual for the Preparation of Graduate Theses'',
%             The Graduate School, Purdue University, 1996.
%      TCMOS  The Chicago Manual of Style, Edition 14.
%      WNNCD  Webster's Ninth New Collegiate Dictionary.
%
%  Lines marked with "%%" may need to be changed.
%

%  % Dedication page is optional.
%  % A name and often a message in tribute to a person or cause.
%  % References: PU 15, WNNCD 332.
%\begin{dedication}
%  To my patient and loving family: Beth, Britton, Lucy, John, and Nora.
%\end{dedication}

  % Abstract is required.
  % Note that the information for the first paragraph of the output
  % doesn't need to be input here...it is put in automatically from
  % information you supplied earlier using \title, \author, \degree,
  % and \majorprof.
  % Reference: PU 17.
\begin{abstract}
  This is the abstract.
\end{abstract}

  % Acknowledgements page is optional but most theses include
  % a brief statement of apreciation or recognition of special
  % assistance.
  % Reference: PU 16.
\begin{acknowledgments}
  The author is grateful to Dr. John Anvik, for his advice and patience, to Dr. Yvonne Chueh for her help with exploratory data analysis, and to Dr. Kathryn Temple for her guidance with time series modeling.
\end{acknowledgments}

%  % The preface is optional.
%  % References: PU 16, TCMOS 1.49, WNNCD 927.
%\begin{preface}
%  This is the preface.
%\end{preface}

  % The Table of Contents is required.
  % The Table of Contents will be automatically created for you
  % using information you supply in
  %     \chapter
  %     \section
  %     \subsection
  %     \subsubsection
  % commands.
  % Reference: PU 16.
\tableofcontents

  % If your thesis has tables, a list of tables is required.
  % The List of Tables will be automatically created for you using
  % information you supply in
  %     \begin{table} ... \end{table}
  % environments.
  % Reference: PU 16.
\listoftables

  % If your thesis has figures, a list of figures is required.
  % The List of Figures will be automatically created for you using
  % information you supply in
  %     \begin{figure} ... \end{figure}
  % environments.
  % Reference: PU 16.
\listoffigures

%  % List of Symbols is optional.
%  % Reference: PU 17.
%\begin{symbols}
%  $m$& mass\cr
%  $v$& velocity\cr
%\end{symbols}

%  % List of Abbreviations is optional.
%  % Reference: PU 17.
%\begin{abbreviations}
%  abbr& abbreviation\cr
%  bcf& billion cubic feet\cr
%  BMOC& big man on campus\cr
%\end{abbreviations}
%
%  % Nomenclature is optional.
%  % Reference: PU 17.
%\begin{nomenclature}
%  Alanine& 2-Aminopropanoic acid\cr
%  Valine& 2-Amino-3-methylbutanoic acid\cr
%\end{nomenclature}
%
%  % Glossary is optional
%  % Reference: PU 17.
%\begin{glossary}
%  chick& female, usually young\cr
%  dude& male, usually young\cr
%\end{glossary}



%
% Put chapter \include commands here.
%

% Introductions may precede the first chapters or major divisions of theses.
% Reference: TM2006, page 31.
% CHANGE NEXT LINE?
\include{introduction}

% Summary and/or conclusions are optional but often used.
% The summary and/or conclusions often are the last
% the last major division(s) of the text.
% Reference: TM2006 page 32.
% CHANGE NEXT LINE?
\include{summary}

% Recommendations are optional.
% You may include recommendations as a major division if your
% subject matter and research dictate.
% Reference: TM2006 page 32.
% CHANGE NEXT LINE?
\include{recommendations}


% Bibliography is required if you consulted any outside references.
% Reference: TM2006 page 32.
\include{bibliography}

% Appendices are optional.
% Appendices are not necessarily a part of every thesis.
% An appendix is used for supplementary illustrative material,
% original data, computer programs, and other material that
% is not necessarily appropriate for inclusion within the
% text of your thesis.
% Reference: TM2006 page 33.
% Use "\appendix" for one appendix or "\appendices" for more than one
% appendix.
% CHANGE NEXT 7 LINES?
\appendices
\include{demo-citations}
\include{demo-figures}
\include{demo-mathematics}
\include{demo-multicols}
\include{demo-tables}
\include{demo-text}


% Notes and footnotes are optional.
% Reference: TM2006 page 34.
% I have not implemented this yet.  Mark Senn 2002-06-03
%%\include{notes}

% A vita is optional for masters theses
% and required for doctoral dissertations.
% Reference: TM2006 page 13.
% CHANGE NEXT LINE?
\include{vita}

\end{document}

% LaTeX won't read after the \end{document} command.
% You can put notes to yourself or LaTeX input not
% ready for use here if you'd like.
